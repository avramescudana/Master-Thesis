\setchapterpreamble[u]{\margintoc}
\chapter{Classical Chromodynamics}
\labch{glasma}

\begin{preview}[]
Following the pedagogical sources \cite{iancucgcintro, iancuphysicsofcgc, muellersmallx, iancunonlinear, taels}, the {\sffamily MV} model is introduced and then employed to obtain an analytic solution for the gluon fields generated by a single nucleus. Using the light-cone quantization of these fields, the gluon occupation factor is computed, revealing the phenomena of gluon saturation.
\end{preview}

\section{The McLerran-Venugopalan model}
McLerran and Venugopalan \cite{mclerven1,mclerven2,mclerven3} derived a classical effective field theory which describes the gluon distribution function for a large nucleus at moderate small-$x$ values.\sidenote{At higher energies, the classical approximation would break and one would need to incorporate quantum effects.}It also provides initial conditions for the quantum evolution of the {\sffamily CGC} towards smaller-$x$ values. 

\subsubsection*{Separation of scales} 
This model is formulated using light-cone coordinates\sidenote{As defined in Equation~(\cref{qcd8}).}and by performing a boost to the infinite momentum frame.\sidenote{In which the momentum of a nucleus moving along the $x^+$ direction is $P^+\rightarrow \infty$.}In this frame, a separation between the small-$x$ and large-$x$ partons exhibits itself.

\vspace{0.5cm}

\begin{figure}[!hbt]
	\centering
    \includesvg[width=0.8\textwidth]{images/diagrams/scales.svg}
    \caption{\normalsize Separation of scales between taken at an arbitrary cut-off $\Lambda$.}
\end{figure}

\begin{note}
    A high-energy nucleus, whose radius may be parametrized as $R_A\approx R_0A^{1/3}$, is Lorentz contracted with
    \begin{equation}\label{glasma2}
        \gamma=\frac{P^+_A}{M_A}=\frac{P^+}{M},
    \end{equation}
    where $P^+_A$ and $M_A$ are the momentum and mass of the nucleus, whereas $M$ is the mass of a single nucleon. Thus, the nucleus is squeezed to
    \begin{equation}\label{glasma3}
        \Delta x^-_A\sim\frac{2R_A}{\gamma}\stackrel{(\text{\cref{glasma2}})}{=\joinrel=\joinrel=}\frac{2R_A}{MP^+}.
    \end{equation}
    Heisenberg's uncertainty principle on the {\sffamily LC} gives the spatial extent of a parton as\sidenote{Where we use the momentum fraction defined as
    \begin{align}{\label{glasma1}}
        x\overset{\Delta}{=}\frac{p^+}{P^+}.
    \end{align}
    }
    \begin{equation}\label{glasma4}
        \Delta x^-\sim\frac{1}{p^+}\stackrel{(\text{\cref{glasma1}})}{=\joinrel=\joinrel=}\frac{1}{xP^+}.
    \end{equation}
    Let us compare Equations~(\cref{glasma3}) and~(\cref{glasma4}). For a large-$x$ parton,\sidenote{For example, valence quarks are large-$x$ partons.}that is $x\gg 1$, $\Delta x^-_{\text{large-}x}\ll \Delta x^-_A$, which means that such a parton is {\sffamily\color{ming}well localized}. On the other hand, a small-$x$ parton\sidenote{Also referred to as wee partons.}with $x\ll 1$ has $\Delta x^-_{\text{small-}x}\gg \Delta x^-_A$\sidenote{This also leads to a condition for how large the small-$x$ variable must be. More precisely, after directly plugging in Equations~(\cref{glasma3}) and~~(\cref{glasma4}), it becomes $x\ll M/2R_A$ but since $R_A\sim A^{1/3}$, this roughly translates to
    \begin{align}\label{glasma9}
        x\ll A^{-1/3}.
    \end{align}
    }, which implies that it is {\sffamily\color{ming}delocalized} along the $x^+$ direction. 
\end{note}

\subsubsection*{Model for the color source} 
With increasing $x$, large-$x$ partons radiate small-$x$ partons. Hence, they act as color sources for the small-$x$ degrees of freedom. Since we chose the nucleus to be moving along the light-cone direction $x^+$, the generated color current will only consists of the component $J^+$, hence $J^\mu\sim\delta^{\mu+}$. \\
We deduced\sidenote{From Equations~(\cref{glasma3}) and~(\cref{glasma4}).}that the small-$x$ partons are delocalized, as opposed to the large-$x$ partons. This implies that the small-$x$ partons see the large-$x$ partons as belonging to an {\sffamily\color{ming}infinitely thin sheet} of color charge, that is $J^\mu\sim\delta(x^-)$. \\
The small-$x$ partons see the large-$x$ partons as {\sffamily\color{ming}static sources} of color charge, that is $J^\mu\neq J^\mu(x^+)$.

\begin{marginfigure}[-5.5cm]
	\centering
    \includesvg[ width=\textwidth]{images/diagrams/lc_current.svg}
    \caption*{Diagram of light-cone current in the {\sffamily MV} model.}
\end{marginfigure}

\begin{note}
    One may also evaluate Heisenberg's uncertainty principle as
    \begin{equation*}
        \Delta x^+\sim\frac{1}{p^-}\stackrel{(\text{\cref{lc37}})}{=\joinrel=\joinrel=}\frac{2p^+}{m_\perp^2}\stackrel{(\text{\cref{glasma1}})}{=\joinrel=\joinrel=}\frac{2xP^+}{p_\perp^2},
    \end{equation*}
    in which we used the light-cone dispersion relation for a massless particle. Hence, $\Delta x^+_{\text{small-}x}\ll \Delta x^+_{\text{large-}x}$, implying that on the timescale of a small-$x$ parton, the large-$x$ partons live considerably longer and thus appear to be frozen in time. Since their momenta are significantly greater, they radiate the small-$x$ partons without recoil.
\end{note}

This particular feature also shows the {\sffamily\color{ming}glassy behaviour} of such a nucleus.\sidenote{It is characteristic of a glass type system to evolve slowly in time, compared to the timescales at which it is being probed.} \\
One now has a model for the light-cone color current generated by the large-$x$ partons: $J^\mu=\delta^{\mu+}\delta(x^-)\rho(\vec{x}_\perp)$, where $\rho(\vec{x}_\perp)$ is the transverse color charge associated to the distribution of these large-$x$ partons. \\

Nevertheless, we are going to relax the assumption of an infinitely thin sheet and write the current as\sidenote{Or on color components
\begin{equation}\label{glasma11}
    J^{\mu,a}=\delta^{\mu+}\rho^a(x^-,\vec{x}_\perp),
\end{equation}
where the color charge is written as $\rho(x^-,\vec{x}_\perp)=\rho^a(x^-,\vec{x}_\perp)t_a$.
}
\coloredeqnum{glasma53}{
    J^\mu=\delta^{\mu+}\rho(x^-,\vec{x}_\perp).
}
In the appropriate range of validity, the large-$x$ partons may be treated as {\sffamily\color{ming}classical random source} of color charge.

\begin{note}
    Let us consider a small-$x$ parton interacting with the nucleus via a virtual photon. In the dipole picture of {\sffamily DIS}, this translates to a quark-antiquark dipole probing the nucleus. If the dipole has a transverse resolution $Q^2$, then it may couple to large-$x$ partons contained in a transverse area $S_\perp=1/Q^2$ of the nucleus.\sidenote{The nucleus having a radius $R_A=R_0A^{1/3}$ with $R_0\sim\Lambda_{\textsf{QCD}}^{-1}$.}The density of probed valence quarks per transverse area would be $n=N_cA/\pi R_A^2$, from which one may derive the total number of probed color charges as
    \begin{align}\label{glasma5}
        \Delta N=n\Delta S_\perp=\frac{N_cA}{\pi R_A^2Q^2}\sim \frac{\Lambda_{\textsf{QCD}}^2}{Q^2}N_cA^{1/3}.
    \end{align}
    The emerging picture is as following: there are many large-$x$ partons\sidenote{If the virtuality of the photon is high enough to penetrate the nucleus, that is $Q^2\gg\Lambda_{\textsf{QCD}}^2$ but low enough 
    \begin{align}\label{glasma10}
        Q^2\ll \Lambda_{\textsf{QCD}}^2N_cA^{1/3},
    \end{align}
    it sees a collection of many color charges $\Delta N \gg 1$.}which act as sources of color charge and belong to different nucleons in a very large nucleus, thus being uncorrelated. Under these circumstances, one may assume that these color sources are {\sffamily\color{ming}distributed randomly} in the transverse plane. Therefore, the average of the total color charge seen by the probe is null\sidenote{On average, the nucleus must be seen as color neutral.}
    \begin{align}\label{glasma6}
        \langle \mathcal{Q}^a \rangle_A =0.
    \end{align}
    The squared color charge of a single quark is $g^2t^at^a=g^2C_f$\sidenote{Here $C_F$ denotes the Casimir operator in the fundamental representation of $\textsf{SU}(3)$, more precisely
    \begin{align*}
        C_F=\frac{N_c^2-1}{2N_c}.
    \end{align*}
    }multiplied with the total number of charges in the transverse plane $\Delta N$ enables us to express the two-point function of total color charges\sidenote{All higher-order averages are assumed to vanish.}as
    \begin{align}\label{glasma7}
        \langle \mathcal{Q}^a\mathcal{Q}^a\rangle_A=g^2C_f\Delta N\stackrel{(\text{\cref{glasma5}})}{=\joinrel=\joinrel=}\frac{C_FN_c}{Q^2}\underbrace{\frac{g^2A}{2\pi R_A^2}}_{\overset{\Delta}{=}\mu_A}.
    \end{align}
    In the last expression we introduced $\mu_A$, the average color charge per transverse area of the valence quarks. Because $\Delta N\gg 1$, we also have $\left(\mathcal{Q}^c\right)^2\gg 1$. Consequently, the commutator
    \begin{align*}
        \left|\left[Q^a,Q^b\right]\right|=\left|if^{abc}Q^c\right|\ll\left(\mathcal{Q}^c\right)^2,
    \end{align*}
    becomes negligible and we may treat the large-$x$ as a {\sffamily\color{ming}classical source} of color charges. These color charges enclosed in a tube of transverse size $S_\perp=1/Q^2$ are generated by continuous a charge distribution $\rho^a\left(x^-,\vec{x}_\perp\right)$ as
    \begin{align}\label{glasma8}
        \mathcal{Q}^a=\int\limits_{1/Q^2}d^2\vec{x}_\perp\underbrace{\int dx^-\rho^a\left(x^-,\vec{x}_\perp\right)}_{\overset{\Delta}{=}\rho^a\left(\vec{x}_\perp\right)}.
    \end{align}
\end{note}

The one-point and two-point correlators\sidenote{They may be derived by inserting Equation~(\cref{glasma8}) in Equations~(\cref{glasma6}) and~(\cref{glasma7}).}encode the main physics of the {\sffamily MV} model

\vspace{0.5cm}

\begin{fullwidth}
\begin{subequations}
\coloredeqnums{
&\langle \rho^a\left(x^-,\vec{x}_\perp\right)\rangle_A=0,\label{glasma33a}\\
&\langle \rho^a\left(x^-,\vec{x}_\perp\right)\rho^b\left(y^-,\vec{y}_\perp\right)\rangle_A=\delta^{ab}\delta\left(x^--y^-\right)\delta^{(2)}\left(\vec{x}_\perp-\vec{y}_\perp\right)\lambda_A(x^-),\label{glasma33b}
}
\end{subequations}
\end{fullwidth}
where $\lambda_A$ represents the average color charge per unit volume.\sidenote{Is is introduced through the relation $\mu_A=\int dx^-\lambda_A(x^-)$.}Higher order non-vanishing correlators may be generated from a Gaussian weight functional\sidenote{Where $\mathcal{N}$ denotes the normalization constant.}
\coloredeq{
\mathcal{W}_A[\rho]=\mathcal{N}\exp{-\frac{1}{2}\int dx^-d^2\vec{x}_\perp \frac{\rho^a(x^-,\vec{x}_\perp)\rho^a(x^-,\vec{x}_\perp)}{\lambda_A(x^-)}},
}
which may be then used to compute the average of an arbitrary observable $\mathcal{O}$ computable from the field $A_\mu$ corresponding to a configuration of the color charge $\rho$ as 
\begin{align*}
    \langle \mathcal{O}[A_\mu]\rangle_A=\frac{\int\mathcal{D}[\rho]\mathcal{W}_A[\rho]\mathcal{O}[A_\mu]}{\int\mathcal{D}[\rho]\mathcal{W}_A[\rho]}.
\end{align*}

\subsubsection*{Range of validity} 
Uncertainty principle on the light-cone constraints the upper bound\sidenote{Deduced in Equation~(\cref{glasma9}).}of the $x$ value for a small-$x$ parton to $x\ll A^{-1/3}$. On the other hand, if the small-$x$ value is too small, the classical picture would not remain valid anymore and one would need to take into account radiative effects.\sidenote{The probability for a small-$x$ parton, that is $x\ll 1$, to radiate a gluon is given by
\begin{align*}
    \int\limits_x^1d\mathcal{P}_{\text{Bremsstrahlung}}\sim \alpha\ln{\frac{1}{x}}.
\end{align*}}This further restricts $\ln{1/x}\ll 1/\alpha$. Therefore, the {\sffamily MV} model is valid in the kinematic range
\begin{align*}
    \ln{A^{1/3}}\ll\ln{1/x}\ll 1/\alpha,
\end{align*}
for a parton with a transverse resolution\sidenote{Showed in Equation~(\cref{glasma10}).}within
\begin{align*}
     \Lambda_{\textsf{QCD}}^2\ll Q^2\ll \Lambda_{\textsf{QCD}}^2N_cA^{1/3}.
\end{align*}



\section{The classic color field}
The classical gluon fields are obtained as solutions of the colored Yang-Mills equations,\sidenote{As given in Equation~(\cref{qcd9}).}with the current of the valence quarks expressed from the {\sffamily MV} model.\sidenote{See Equation~(\cref{glasma11}).}This yields
\begin{align}\label{glasma13}
    \big(\textsf{D}_\nu F^{\nu\mu}\big)(x^-,\vec{x}_\perp)=\delta^{\mu+}\rho(x^-,\vec{x}_\perp).
\end{align}

The fields $A^+$ and $A^i$ must be independent of the light-cone time,\sidenote{Since the charge distribution $\rho$ is static, it is fair to assume that this property is passed to the fields generated by this distribution.}that is\sidenote{Or equivalently
\begin{equation}\label{glasma14}
    \partial^+A^+=\partial^+A^i=0.
\end{equation}
}
\begin{equation*}
    \partial^-A^+=\partial^-A^i=0.
\end{equation*}

Moreover, in order for the light-cone current to be covariantly conserved,\sidenote{This is not automatically assured since 
\begin{equation*}
    \textsf{D}_+J^+=\partial_+\rho-ig[A^-,\rho],
\end{equation*}
with $\partial_+\rho=0$, because the current is static, but generally $[A^-,\rho]\neq0$.
}that is $\textsf{D}_+J^+=0$, one must also impose
\begin{equation}\label{glasma12}
    A^-=0.
\end{equation}

This further leads to $\textsf{D}_jF^{ji}=0$.


% \begin{subequations}
%     \begin{align}
%         &F^{ij}=0,\label{glasma12a}\\
%         &A^-=0,\label{glasma12b}\\
%         &\partial_+A^+=\partial_+A^i=0\label{glasma12c}.
%     \end{align}
% \end{subequations}

\begin{note}
    The components of the field strength tensor become
    \begin{subequations}
    %\label{e47}
        \begin{align}
        &F^{+-}\underset{(\text{\cref{glasma12}})}{\stackrel{(\text{\cref{glasma14}})}{=\joinrel=\joinrel=\joinrel=}}\partial^+\cancelto{0}{A^-}-\cancelto{0}{\partial^-A^+}-ig[A^+,\cancelto{0}{A^-}]&&=0,\label{glasma15a}\\
        &F^{i-}\underset{(\text{\cref{glasma12}})}{\stackrel{(\text{\cref{glasma14}})}{=\joinrel=\joinrel=\joinrel=}}\partial^i\cancelto{0}{A^-}-\cancelto{0}{\partial^-A^i}-ig[A^i,\cancelto{0}{A^-}]&&=0,\label{glasma15b}\\
        &F^{i+}=\partial^iA^+-\partial^+A^i-ig[A^i,A^+].\label{glasma15c}
     \end{align}
    \end{subequations}
    Equation~(\cref{glasma13}) for the $\mu=-$ component\sidenote{For the $\mu=+$, it becomes an equation for the only non-vanishing field strength, that is
    \begin{equation}\label{glasma17}
        \textsf{D}_-\cancelto{0}{F^{-+}}+\textsf{D}_iF^{i+}=\rho.
    \end{equation}
    }is immediately satisfied
    \begin{equation*}
        \textsf{D}_+\cancelto{0}{F^{+-}}+\textsf{D}_i\cancelto{0}{F^{i-}}\underset{(\text{\cref{glasma15b}})}{\stackrel{(\text{\cref{glasma15a}})}{=\joinrel=\joinrel=\joinrel=}}0,
    \end{equation*}
    whereas for the $\mu=i$ it yields
    \begin{equation*}
        \textsf{D}_+F^{+i}-\textsf{D}_-\cancelto{0}{F^{-i}}+\textsf{D}_jF^{ji}(\text{\cref{glasma15b}})\stackrel{(\text{\cref{glasma15b}})}{=\joinrel=\joinrel=\joinrel=} \textsf{D}_+F^{+i}+\textsf{D}_jF^{ji}.
    \end{equation*}
    Since simple manipulations allow us to write
    \begin{equation*}
        \begin{aligned}
        \textsf{D}_+F^{+i}&\stackrel{(\text{\cref{glasma12}})}{=\joinrel=\joinrel=\joinrel=}(\partial^--ig\cancelto{0}{A^-})F^{+i}\\
        &\stackrel{(\text{\cref{glasma15c}})}{=\joinrel=\joinrel=\joinrel=}\partial^-\big(\partial^iA^+-\partial^+A^i-ig[A^i,A^+]\big)\stackrel{(\text{\cref{glasma14}})}{=\joinrel=\joinrel=\joinrel=}0,
        \end{aligned}
    \end{equation*}
    which immediately leads to $\textsf{D}_jF^{ji}=0$.
\end{note}

Therefore $F^{ij}=0$, satisfied by a $A^i$ which is a {\sffamily\color{ming}pure gauge field} in the transverse plane. Such a field may be written as\sidenote{According to Equation~(\cref{qcd10}).}
\coloredeqnum{glasma16}{
    A^i(x^-,\vec{x}_\perp)=\frac{i}{g}\textsf{W}(x^-,\vec{x}_\perp)\partial^i\textsf{W}^\dag(x^-,\vec{x}_\perp).
}
With these choices, the only remaining independent fields are $A^+$ and $A^i$. On top of that, one may also fix the gauge, which further reduces the number of field degrees of freedom.


\subsubsection*{Solution in the covariant gauge} 
Since $A^i$ is a pure gauge field, it may be gauge transformed to become null,\sidenote{This may be checked by simply gauge transforming the pure gauge field as
\begin{equation*}
    \begin{aligned}
    &\widetilde{A}^i\stackrel{(\text{\cref{qcd11}})}{=\joinrel=\joinrel=}\textsf{U}^\dag A^i \textsf{U}+\frac{i}{g}\textsf{U}^\dag\left(\partial^i \textsf{U}\right)\\
    &\stackrel{(\text{\cref{glasma16}})}{=\joinrel=\joinrel=}\frac{i}{g}\big[\underbrace{\textsf{U}^\dag\textsf{U}}_{\mathds{1}}\big(\partial^i\textsf{U}^\dag\big)\textsf{U}+\textsf{U}^\dag\left(\partial^i\textsf{U}\right)\big]\\
    &=\frac{i}{g}\big[\cancelto{0}{\partial^i\big(\underbrace{\textsf{U}^\dag\textsf{U}}_{\mathds{1}}\big)}-\cancel{\textsf{U}^\dag\left(\partial^i\textsf{U}\right)}+\\
    &\phantom{=}+\cancel{\textsf{U}^\dag\left(\partial^i\textsf{U}\right)}\big]=0.
    \end{aligned}
\end{equation*}
}that is 
\begin{equation}\label{glasma18}
    \widetilde{A}^i=0.
\end{equation}
From this, it follows that $\partial_i\widetilde{A}^i=0$. We previously had $\partial_+A^+=0$, which also remains valid in this gauge $\partial_+\widetilde{A}^+=0$. Since $A^-=0$, we will still have $\widetilde{A}^-=0$ and hence $\partial_-\widetilde{A}^-=0$. Summarizing, the fields in this particular gauge satisfy $\partial_\mu\widetilde{A}^\mu=0$, which is the condition for the {\color{ming}\sffamily covariant gauge}. \\
In this gauge, the only remaining field strength tensor further simplifies to
\begin{equation}\label{glasma19}
    \widetilde{F}^{i+}\underset{(\text{\cref{glasma18}})}{\stackrel{(\text{\cref{glasma15c}})}{=\joinrel=\joinrel=\joinrel=}}\partial^i\widetilde{A}^+-\partial^+\cancelto{0}{\widetilde{A}^i}-ig[\cancelto{0}{\widetilde{A}^i},\widetilde{A}^+],
\end{equation}
which after introducing the notation\sidenote{A notation borrowed from the non-Abelian case.}
\begin{align*}
    \widetilde{A}^+(x^-,\vec{x}_\perp)\overset{\Delta}{=}\alpha(x^-,\vec{x}_\perp),
\end{align*}
leads to a Poisson equation in the transverse plane for the gauge field\sidenote{The LHS of Equation~(\cref{glasma17}) in the covariant gauge becomes
\begin{equation*}
    (\partial_i-ig\cancelto{0}{\widetilde{A}^i})\widetilde{F}^{i+}\stackrel{(\text{\cref{glasma19}})}{=\joinrel=\joinrel=}\underbrace{\partial_i\partial^i}_{-\nabla^2_\perp}\alpha,
\end{equation*}
where $\nabla^2_\perp$ represents the transverse Laplace operator $\Delta_\perp$.
}
\coloredeqnum{glasma20}{
    \Delta_\perp\alpha(x^-,\vec{x}_\perp)=-\widetilde{\rho}(x^-,\vec{x}_\perp),
}
where $\widetilde{\rho}$ is the color charge in the covariant gauge.\sidenote{Which is related to the color charge in the light-cone gauge through
\begin{equation}\label{glasma34}
    \widetilde{\rho}=\textsf{W}^\dagger\rho\textsf{W}.
\end{equation}
}This equation has the following solution
\begin{equation*}
    \alpha(x^-,\vec{x}_\perp)=-\int d^2\vec{y}_\perp\textsf{G}_\perp(\vec{x}_\perp-\vec{y}_\perp)\rho(x^-,\vec{y}_\perp),
\end{equation*}
in which we introduced the Green function for the transverse Laplace operator as\sidenote{It is also necessary to introduce an infrared cutoff $\Lambda$.}
\begin{equation}\label{glasma21}
    \begin{aligned}
    \textsf{G}_\perp(\vec{x}_\perp-\vec{y}_\perp)&\overset{\Delta}{=}\Braket{\vec{x}_\perp|\frac{1}{\nabla^2_\perp}|\vec{y}_\perp}\\
    &=\frac{1}{4\pi}\ln{\frac{1}{\left(\vec{x}_\perp-\vec{y}_\perp\right)^2\Lambda^2}}.
    \end{aligned}
\end{equation}

\begin{note}
    The solution of Equation~(\cref{glasma20}) may formally be written in the Fourier space as
    \begin{equation*}
        \alpha(x^-,\vec{x}_\perp)=-\int d^2\vec{y}_\perp\int \frac{d^2k_\perp}{(2\pi)^2}\frac{\rho(x^-,\vec{y}_\perp)}{k_\perp^2}\mathrm{e}^{i\vec{k}_\perp\left(\vec{x}_\perp-\vec{y}_\perp\right)}.
    \end{equation*}
    Let us perform a change of variables $\vec{r}_\perp\overset{\Delta}{=}\vec{x}_\perp-\vec{y}_\perp$, which yields
    \begin{align*}
    \alpha(x^-,\vec{x}_\perp)=-\int d^2\vec{r}_\perp\underbrace{\int \frac{d^2k_\perp}{(2\pi)^2}\frac{1}{k_\perp^2}\mathrm{e}^{i\vec{k}_\perp\vec{r}_\perp}}_{\overset{\Delta}{=}\textsf{G}_\perp(\vec{r}_\perp)}\rho(x^-,\vec{x}_\perp-\vec{r}_\perp).
    \end{align*}
    At this point, one may easily recognise the Green function for the two-dimensional Laplace equation.\sidenote{Which is just the definition from Equation~(\cref{glasma21}) written explicitly as a Fourier transform.}The infrared divergence that appears in this expression may be artificially eliminated by considering an inferior limit to the integration, namely $\Lambda$, which then leads to\sidenote{Where $\textsf{J}_0$ is the Bessel function of first kind and order zero.}
    \begin{fullwidth}
    \begin{equation*}
        \begin{aligned}
        \textsf{G}_\perp(\vec{r}_\perp)&=\frac{1}{(2\pi)^2}\int\limits_\Lambda^\infty\int\limits_0^{2\pi}k_\perp dk_\perp d\theta\frac{1}{k_\perp^2}\mathrm{e}^{ik_\perp r \cos\theta}=\frac{1}{2\pi}\int\limits_\Lambda^\infty\frac{dk_\perp}{k_\perp}\underbrace{\frac{1}{2\pi}\int\limits_0^{2\pi}d\theta \mathrm{e}^{ik_\perp r\cos\theta}}_{\textsf{J}_0(k_\perp r)}=\frac{1}{4\pi}\ln{\frac{1}{r_\perp^2\Lambda^2}}.
        \end{aligned}
    \end{equation*}    
    \end{fullwidth}
\end{note}


\subsubsection*{Solution in the light-cone gauge} 
Since we intent to use the light-cone quantization of the fields, let us perform a gauge transformation\sidenote{We shall denote it as $\textsf{W}(x^-,\vec{x}_\perp)\in\textsf{SU}(N_c)$.}of the gauge field from the covariant gauge $\widetilde{A}^i=0$ to the light-cone gauge $A^+=0$ as
\begin{equation*}
    A^\mu=\textsf{W}^\dag\left(\widetilde{A}^\mu+\frac{i}{g}\partial^\mu\right)\textsf{W}.
\end{equation*}
This gives\sidenote{For the $\mu=+$ component.}an equation for the gauge transformation
\begin{equation*}
    \textsf{W}^\dag\left(\alpha+\frac{i}{g}\partial^+\right)\textsf{W}=0,
\end{equation*}
which one may recognise as the {\color{ming}\sffamily parallel transport equation for a Wilson line} along a given path, whose solution is a path ordered exponential\sidenote{The lower limit of the integral was chosen to assure the retarded boundary condition
\begin{align*}
    A^i(x^-,\vec{x}_\perp)\xrightarrow[]{x^-\rightarrow\infty}0.
\end{align*}
} 
\coloredeqnum{glasma52}{
    \textsf{W}^\dag(x^-,\vec{x}_\perp)=\mathcal{P}\exp{ig\int\limits_{-\infty}^{x^-}dz^-\alpha(z^-,\vec{x}_\perp)}.
}

The gauge transformation\sidenote{Written for $\mu=i$.} provides the solution in the light-cone gauge,\sidenote{As already guessed in Equation~(\cref{glasma16}).}which looks like a transverse pure gauge rotated in the color space.\sidenote{In the literature, it is commonly referred to as the non-Abelian Weizsacker-Williams field \cite{kovchegov}.}The corresponding field strength in the light-cone gauge is given by\sidenote{Or equivalently in momentum space
\begin{equation}\label{glasma23}
    F^{i+}(k)=ik^+A^i(k).
\end{equation}
}
\begin{align*}
F^{i+}=\partial^i\cancelto{0}{A^+}-\partial^+A^i-ig[A^i, \cancelto{0}{A^+}].
\end{align*}

Nevertheless, it is more useful to express it in terms of the field $\alpha$ in the covariant gauge.\sidenote{Which is a solution of Equation~(\cref{glasma20}).}This may achieved by performing a gauge transforming the field strength from the covariant gauge\sidenote{As written in Equation~(\cref{glasma19})}as
\begin{align}\label{glasma30}
    F^{i+}=\textsf{W}^\dagger\widetilde{F}^{i+}\textsf{W}.
\end{align}
On color components $F^{i+}=F^{i+}_at^a$, this yields\sidenote{Also for the covariant solution $\alpha(x^-,\vec{x}_\perp)=\alpha_a(x^-,\vec{x}_\perp)t^a$}.
\coloredeqnum{glasma31}{
    F^{i+}_a(x^-,\vec{x}_\perp)=\textsf{W}_{ba}(x^-,\vec{x}_\perp)\partial^i\alpha_b(x^-,\vec{x}_\perp).
}
\begin{note}
    Simple manipulations of already obtained results give
    \begin{align*}
        F^{i+}\underset{(\text{\cref{glasma19}})}{\stackrel{(\text{\cref{glasma30}})}{=\joinrel=\joinrel=}}\textsf{W}^\dagger(\partial^i\underbrace{\alpha_bt^b}_{\alpha})\textsf{W}=\underbrace{\textsf{W}^\dagger t^b\textsf{W}}_{\textsf{W}_{ba}t^a}\partial^i\alpha_b=\underbrace{\textsf{W}_{ba}\partial^i\alpha_b}_{F^{i+}_a}t^a.
    \end{align*}
\end{note}

\section{Gluon saturation from the {\sffamily MV model}}
\subsubsection*{Computing observables}
The {\sffamily MV} model along with {\sffamily LC} quantization provide a framework for computing the gluon distribution function, as probed in {\sffamily DIS} experiments. 

\subsubsection*{Light-cone gluon distribution}
A parton which carries a momentum fraction $x=k^+/P^+$ may probe the distribution of gluons $xG(x,Q^2)$, defined as the number of gluons contained in a transverse plane\sidenote{Or having momenta $k_\perp<Q$. This will introduce the step function $\Theta$.}of size $\Delta S_\perp\sim Q^2$
\begin{fullwidth}
\begin{equation}\label{glasma25}
    x G\left(x, Q^{2}\right) =\left.\int\limits^{Q^{2}} \mathrm{d}^{2} \vec{k}_{\perp} k^{+} \frac{\mathrm{d} N}{\mathrm{d} k^{+} \mathrm{d}^{2} \vec{k}_\perp}\right|_{k^{+}=x P^{+}}=\int \mathrm{d}^{3} k \Theta\left(Q^{2}-k_{\perp}^{2}\right) x \delta\left(x-\frac{k^{+}}{P^{+}}\right) \frac{\mathrm{d} N}{\mathrm{d}^{3} k}.
\end{equation}
\end{fullwidth}
In the {\sffamily MV} model, the transverse gluon fields\sidenote{See Equation~(\cref{glasma16}).}may be quantized in the light-cone gauge as\sidenote{Following the {\sffamily LC} quantization, as in Equation~(\cref{lc34}). We introduce the notation $\vec{x}{\overset{\Delta}{=}}(x^-,\vec{x}_\perp)$.}
\begin{equation*}
    A_a^{i}\left(x^{+}, \vec{x}\right)=\int\limits_{k^{+}>0} \frac{d^{3} k}{(2 \pi)^{3} 2 k^{+}}\left[a_a^{i}(x^{+}, \vec{k})e^{i \vec{k} \cdot \vec{x}} + a_a^{i \dagger}(x^{+}, \vec{k})e^{-i \vec{k} \cdot \vec{x}}\right],
\end{equation*}
with the creation and annihilation operators obeying equal light-cone time commutation relations.\sidenote{See Equation~(\cref{lc42}).}These enables us to explicitly write the density of gluons in the Fock space as\sidenote{The averages are taken over the hadron wavefunction.}
\begin{equation}\label{glasma24}
\begin{aligned}
    \frac{dN}{d^3k}&=\langle a^{i\dag}_a(x^+,\vec{k})a^i_a(x^+,\vec{k})\rangle\\
    &=\frac{2 k^{+}}{(2 \pi)^{3}} \big\langle A_a^i(x^{+}, \vec{k}) A_a^i(x^{+},-\vec{k})\big\rangle.
\end{aligned}
\end{equation}
Since in the light-cone gauge the only non-vanishing field strength $F^{i+}$ is linearly related to the $A^i$ gauge fields,\sidenote{As in Equation~(\cref{glasma23}).}the gluon distribution function may further be expressed as\sidenote{By replacing Equation~(\cref{glasma25}) back in Equation~(\cref{glasma24}) and making use of the fact that the fields are independent of the light-cone time. For this reason, we shall omit it in the following computations.}
\begin{equation}\label{glasma26}
    x G\left(x, Q^{2}\right)= \int \frac{d^{2} k_{\perp}}{4\pi^3} \Theta(Q^2-k_\perp^2)\langle F_{a}^{i+}(\vec{k}) F_{a}^{i+}(-\vec{k})\rangle.
\end{equation}

\subsubsection*{Gauge invariance}
The above expression is not gauge invariant, which may be easily deduced by looking at the expression contained in the two-point function
\begin{fullwidth}
\begin{align}\label{glasma32}
    F_{a}^{i+}(\vec{k}) F_{a}^{i+}(-\vec{k})=\int \mathrm{d}^{3} x \int \mathrm{d}^{3} y \text{  }\mathrm{e}^{i(\vec{x}-\vec{y}) \cdot \vec{k}} \textsf{Tr}\{F^{i+}(\vec{x})F^{i+}(\vec{y})\}.
\end{align}
\end{fullwidth}
Nevertheless, it may be given a gauge invariant\sidenote{Written in coordinate space, it involves field strengths at different points $\vec{x}$ and $\vec{y}$ and thus is not gauge invariant.}significance through an appropriate choice of gauge, path and boundary conditions for the gauge fields. In general, one may construct from $\textsf{Tr}\{F^{i+}(\vec{x})F^{i+}(\vec{y})\}$ a gauge invariant operator as\sidenote{The path $\gamma$ is oriented from $\vec{x}$ to $\vec{y}$.} 
\begin{equation}\label{glasma22}
    \textsf{Tr}\left\{F^{i+}(\vec{x}) \textsf{U}_{\gamma}(\vec{x}, \vec{y}) F^{i+}(\vec{y}) \textsf{U}_{\gamma}(\vec{y}, \vec{x})\right\}.
\end{equation}
by simply inserting Wilson lines\sidenote{With $\vec{A}{\overset{\Delta}{=}}(A^+,\vec{A}_\perp)$.}
\begin{align*}
    \textsf{U}_{\gamma}(\vec{x}, \vec{y})=\mathcal{P} \exp{i g \int\limits_{\gamma} d \vec{z} \cdot \vec{A}(z^-,\vec{z}_\perp)}.
\end{align*}

\begin{figure}[!hbt]
	\centering
    \includesvg[width=\textwidth]{images/diagrams/path.svg}
    \caption{\normalsize Path chosen such that the gluon distribution function will have a gauge invariant meaning.}
\end{figure}

For a specific choice of the path, using the light-cone gauge $A^+=0$ and by imposing retarded boundary conditions\sidenote{With respect to $x^-$.} 
\begin{align*}
    A^i(x^-,\vec{x}_\perp)\xrightarrow[]{x^-\rightarrow\infty}0,
\end{align*}
the term from the exponent of the Wilson line gives a null contribution
\begin{fullwidth}
\begin{equation*}
    \int\limits_{\gamma}d \vec{z} \cdot \vec{A}(z^-,\vec{z}_\perp)=\int\limits_{y^-}^{-\infty}dz^-\cancelto{0}{A^+(z^-,\vec{y}_\perp)}+\int\limits_{\vec{y}_\perp}^{\vec{x}_\perp}d^2\vec{z}_\perp\cancelto{0}{\vec{A}_\perp(-\infty,\vec{z}_\perp)}+\int\limits_{-\infty}^{x^-}dz^-\cancelto{0}{A^+(z^-,\vec{x}_\perp)}.
\end{equation*}
\end{fullwidth}
This leads to $\textsf{U}_{\gamma}(\vec{x}, \vec{y})\rightarrow \mathds{1}$ and thus the two-point function correlator which enters in the expression for the gluon distribution becomes gauge invariant.\sidenote{The gauge invariant operator from Equation~(\cref{glasma22}) reduces to $\textsf{Tr}\{F^{i+}(\vec{x})F^{i+}(\vec{y})\}$.}


\subsubsection*{Gluon occupation number}
Let us define the gluon distribution in the transverse plane as\sidenote{The number of gluons per unit of rapidity $y\overset{\Delta}{=}\ln{1/x}$ per unit of transverse momentum $k_\perp$ per unit of transverse area given by the impact parameter $b_\perp$.}
\begin{equation}\label{glasma28}
    n_y(k_\perp,b_\perp)\overset{\Delta}{=}\frac{d^5N}{dyd^2k_\perp d^2b_\perp}=\frac{d^2xG(x,k_\perp^2)}{d^2k_\perp d^2b_\perp}.
\end{equation}
In the case of a nucleus which is homogeneous in transverse plane, the gluon density further simplifies to
\begin{align}\label{glasma27}
    n_y(k_\perp)\approx\frac{xG(x,Q^2)}{S_\perp}\stackrel{(\text{\cref{glasma26}})}{=\joinrel=\joinrel=}\frac{\langle F_{a}^{i+}(\vec{k}) F_{a}^{i+}(-\vec{k})\rangle}{4\pi^3S_\perp}.
\end{align}
We may construct a dimensionless quantity which encodes the gluon overlapping as\sidenote{The number of gluons with a certain spin and color per unit rapidity and per unit of transverse phase-space.}
\begin{align*}
    f_y(k_\perp,b_\perp)\overset{\Delta}{=}\frac{(2\pi)^3}{2(N_c^2-1)}\frac{d^5N}{dyd^2k_\perp d^2b_\perp}\stackrel{(\text{\cref{glasma28}})}{=\joinrel=\joinrel=}\frac{(2\pi)^3n_y(k_\perp,b_\perp)}{2(N_c^2-1)},
\end{align*}
which in the case of a transversely homogeneous nucleus reduces to
\coloredeqnum{glasma29}{
    f_y(k_\perp)\approx\frac{(2\pi)^3n_y(k_\perp)}{2(N_c^2-1)}\stackrel{(\text{\cref{glasma27}})}{=\joinrel=\joinrel=}\frac{\langle F_{a}^{i+}(\vec{k}) F_{a}^{i+}(-\vec{k})\rangle}{(N_c^2-1)\pi S_\perp}.
}

\subsubsection*{Gluon saturation}
Exact expressions for the gluon distribution\sidenote{From Equation~(\cref{glasma26})}and gluon occupation number\sidenote{As defined in Equation~(\cref{glasma29})}may analytically be computed in the {\sffamily MV} model. Both involve the two-point function\sidenote{See Equation~(\cref{glasma32})}$\langle F^{i+}_a(\vec{x})F^{i+}_a(\vec{y})\rangle_A$. It is more suitable to express it as a correlator of fields in the covariant gauge\sidenote{By using Equation~(\cref{glasma31})}
\begin{align}\label{glasma38}
    \langle F^{i+}_a(\vec{x})F^{i+}_a(\vec{y})\rangle_A=\langle(\textsf{W}_{a b} \partial^{i} \alpha_b)({\vec{x}})(\textsf{W}_{a c} \partial^{i} \alpha_c)({\vec{y}})\rangle_A.
\end{align}
This expectation value may be evaluated by making use of 
\vspace*{-\baselineskip}
\begin{fullwidth}
\begin{subequations}
\coloredeqnums{
    &\langle \alpha^a\left(x^-,\vec{x}_\perp\right)\rangle_A=0,\label{glasma35a}\\
    &\langle \alpha^a\left(x^-,\vec{x}_\perp\right)\alpha^b\left(y^-,\vec{y}_\perp\right)\rangle_A=\delta^{ab}\delta\left(x^--y^-\right)\lambda_A(x^-)\textsf{L}(\vec{x}_\perp-\vec{y}_\perp),\label{glasma35b}
}
\end{subequations}
\end{fullwidth}
where we introduced 
\begin{align}\label{glasma37}
    \textsf{L}(\vec{x}_\perp-\vec{y}_\perp)\overset{\Delta}{=}\int \frac{\mathrm{d}^{2} \vec{k}_{\perp}}{(2 \pi)^{2}} \frac{\mathrm{e}^{i \vec{k}_{\perp}(\vec{x}_\perp-\vec{y}_\perp)}}{k_{\perp}^{4}}.
\end{align}

\begin{note}
    First, let us emphasize that the correlators of color charge in the light-cone gauge\sidenote{As given in Equations~(\cref{glasma33a}) and~(\cref{glasma33b}).}are also valid for the color charge in the covariant gauge, if expressed in terms of the light-cone color charge\sidenote{See Equation~(\cref{glasma21}).}
    \begin{fullwidth}
    \begin{subequations}
    \begin{align}
        &\langle \widetilde{\rho}^a\left(x^-,\vec{x}_\perp\right)\rangle_A=0,\label{glasma36a}\\
        &\langle \widetilde{\rho}^a\left(x^-,\vec{x}_\perp\right)\widetilde{\rho}^b\left(y^-,\vec{y}_\perp\right)\rangle_A=\delta^{ab}\delta\left(x^--y^-\right)\delta^{(2)}\left(\vec{x}_\perp-\vec{y}_\perp\right)\lambda_A(x^-).\label{glasma36b}
    \end{align}
    \end{subequations}
    \end{fullwidth}
    The vanishing expectation value of the covariant gauge field\sidenote{From Equation~(\cref{glasma35a})}is a direct consequence of
    \begin{align*}
        &\langle \alpha^a(x^-,\vec{x}_\perp)\rangle_A\stackrel{(\text{\cref{glasma20}})}{=\joinrel=\joinrel=}\left\langle\frac{\widetilde{\rho}^a(x^-,\vec{x}_\perp)}{-\nabla_\perp^2}\right\rangle_A\\
        &\stackrel{(\text{\cref{glasma36a}})}{=\joinrel=\joinrel=\joinrel=}-\int d^2\vec{y}_\perp\int \frac{d^2\vec{k}_\perp}{(2\pi)^2}\frac{\cancelto{0}{\langle\widetilde{\rho}(x^-,\vec{y}_\perp)\rangle_A}}{k_\perp^2}\mathrm{e}^{i\vec{k}_\perp\left(\vec{x}_\perp-\vec{y}_\perp\right)}.
    \end{align*}
    In a similar manner, the two-point correlator\sidenote{From Equation~(\cref{glasma35b})}may be derived as
    \begin{fullwidth}
    \begin{align*}
        &\langle \alpha^a(x^-,\vec{x}_\perp)\alpha^b(y^-,\vec{y}_\perp)\rangle_A\stackrel{(\text{\cref{glasma20}})}{=\joinrel=\joinrel=}\left\langle\frac{\widetilde{\rho}^a(x^-,\vec{x}_\perp)}{-\nabla_\perp^2}\frac{\widetilde{\rho}^b(y^-,\vec{y}_\perp)}{-\nabla_\perp^2}\right\rangle_A\\
        &\stackrel{(\text{\cref{glasma36b}})}{=\joinrel=\joinrel=\joinrel=}\int d^2\vec{x}^\prime_\perp\int \frac{d^2\vec{k}_\perp}{(2\pi)^2}\frac{\mathrm{e}^{i\vec{k}_\perp\left(\vec{x}_\perp-\vec{x}^\prime_\perp\right)}}{k_\perp^2}\int d^2\vec{y}^\prime_\perp\int \frac{d^2\vec{k}^\prime_\perp}{(2\pi)^2}\frac{\mathrm{e}^{i\vec{k}^\prime_\perp\left(\vec{y}_\perp-\vec{y}^\prime_\perp\right)}}{k^{\prime 2}_\perp}\underbrace{\langle \widetilde{\rho}^a\left(x^-,\vec{x}^\prime_\perp\right),\widetilde{\rho}^b\left(y^-,\vec{y}^\prime_\perp\right)\rangle_A}_{\delta^{ab}\delta\left(x^--y^-\right)\delta^{(2)}\left(\vec{x}^\prime_\perp-\vec{y}^\prime_\perp\right)\lambda_A(x^-)}\\
        &=\delta^{ab}\delta\left(x^--y^-\right)\lambda_A(x^-)\underbrace{\int d^2\vec{x}^\prime_\perp \mathrm{e}^{-i\vec{x}^\prime_\perp(\vec{k}_\perp+\vec{k}^\prime_\perp)}}_{\cancel{(2\pi)^2}\delta^{(2)}(\vec{k}_\perp+\vec{k}^\prime_\perp)}\int \frac{d^2\vec{k}_\perp}{\cancel{(2\pi)^2}}\frac{\mathrm{e}^{i\vec{k}_\perp\vec{x}_\perp}}{\vec{k}_\perp^2}\int \frac{d^2\vec{k}^\prime_\perp}{(2\pi)^2}\frac{\mathrm{e}^{i\vec{k}^\prime_\perp\vec{y}_\perp}}{\vec{k}^{\prime 2}_\perp}\\
        &\stackrel{(\text{\cref{glasma37}})}{=\joinrel=\joinrel=\joinrel=}\delta^{ab}\delta\left(x^--y^-\right)\lambda_A(x^-)\underbrace{\int \frac{\mathrm{d}^{2} \vec{k}_{\perp}}{(2 \pi)^{2}} \frac{\mathrm{e}^{i \vec{k}_{\perp}(\vec{x}_\perp-\vec{y}_\perp)}}{k_{\perp}^{4}}}_{\textsf{L}(\vec{x}_\perp-\vec{y}_\perp)}.
    \end{align*}
    \end{fullwidth}
\end{note}

Since the two-point field correlator\sidenote{From Equation~(\cref{glasma38}).}contains only terms local in $x^-$, it may be expressed as
\begin{fullwidth}
\begin{equation}\label{glasma40}
   \begin{aligned}[t]
    \langle F^{i+}_a(\vec{x})F^{i+}_a(\vec{y})\rangle_A=\langle\partial^{i} \alpha_{b}(\vec{x}) \partial^{i} \alpha_{c}(\vec{y})\rangle_A \langle \textsf{W}_{a b}(\vec{x}) \textsf{W}_{ac}(\vec{y})\rangle_A.
\end{aligned} 
\end{equation}
\end{fullwidth}
The first term may further be evaluated as\sidenote{We introduce the notation $\vec{r}_\perp\overset{\Delta}{=}\vec{x}_\perp-\vec{y}_\perp$.}
\begin{fullwidth}
\begin{align}\label{glasma39}
    \langle\partial^{i} \alpha_{b}(x^-,\vec{x}_\perp) \partial^{i} \alpha_{c}(y^-,\vec{y}_\perp)\rangle_A=\delta_{bc}\delta(x^--y^-)\frac{\partial\mu_A(x^-)}{\partial x^-}\frac{1}{4\pi}\ln{\frac{1}{\vec{r}_\perp^2\Lambda_{\textsf{QCD}}^2}}.
\end{align}
\end{fullwidth}


\begin{note}
    Following the same type of computation,\sidenote{Writing the fields in terms of the color charges from Equation~(\cref{glasma20}), working in the Fourier space and using the correlator of such charges from Equation~(\cref{glasma36b})}such a term would yield
    \begin{align*}
        \langle\partial^{i} \alpha_{b}(\vec{x}) \partial^{i} \alpha_{c}(\vec{y})\rangle_A&\stackrel{(\text{\cref{glasma37}})}{=\joinrel=\joinrel}\delta_{bc}\delta\left(x^--y^-\right)\lambda_A(x^-)\times\\
        &\phantom{\stackrel{(\text{\cref{glasma37}})}{=\joinrel=\joinrel}}\times\underbrace{\int \frac{\mathrm{d}^{2} \vec{k}_{\perp}}{(2 \pi)^{2}} \frac{\mathrm{e}^{i \vec{k}_{\perp}(\vec{x}_\perp-\vec{y}_\perp)}}{k_{\perp}^2}}_{-\nabla_\perp^2\textsf{L}(\vec{x}_\perp-\vec{y}_\perp)}.
    \end{align*}
    Nevertheless, we recognise the already computed\sidenote{See Equation~(\cref{glasma21}), from which it follows that
    \begin{align*}
        -\nabla_\perp^2\textsf{L}(\vec{r})=\textsf{G}_\perp(\vec{r}_\perp).
    \end{align*}
    }Green function for the two-dimensional Laplace operator.\sidenote{With $\Lambda=\Lambda_{\textsf{QCD}}$.}By rewriting the color charge density,\sidenote{The color charge density per transverse area may be obtained from the color charge density as
    \begin{equation}\label{glasma45}
        \mu_A(x^-)=\int\limits_{-\infty}^{x^-}dz^-\lambda_A(z^-).
    \end{equation}
    }we obtain
    \begin{align*}
        \langle\partial^{i} \alpha_{b}(\vec{x}) \partial^{i} \alpha_{c}(\vec{y})\rangle_A\stackrel{(\text{\cref{glasma21}})}{=\joinrel=\joinrel=}\delta_{bc}\delta(x^--y^-)\underbrace{\frac{\partial\mu_A(x^-)}{\partial x^-}}_{\lambda_A(x^-)}\frac{1}{4\pi}\ln{\frac{1}{r_\perp^2\Lambda_{\textsf{QCD}}^2}}.
    \end{align*}
\end{note}

This further leads to\sidenote{By inserting the result from Equation~(\cref{glasma39}) back in Equation~(\cref{glasma40}).} 
\begin{equation}\label{glasma48}
    \begin{aligned}
    \langle F^{i+}_a(\vec{x})F^{i+}_a(\vec{y})\rangle_A&=\frac{\partial\mu_A(x^-)}{\partial x^-}\frac{1}{4\pi}\ln{\frac{1}{r_\perp^2\Lambda_{\textsf{QCD}}^2}}\times\\
    &\phantom{=}\times\langle \textsf{W}_{ab}(x^-,\vec{x}_\perp) \textsf{W}_{ba}^\dagger(x^-,\vec{y}_\perp)\rangle_A.
    \end{aligned}
\end{equation}
One may now recognise, up to a color factor, the dipole operator in the adjoint representation

\vspace{0.2cm}

\begin{fullwidth}
\coloredeqnum{glasma41}{
    \textsf{D}_A(x^-,\vec{r}_\perp) \overset{\Delta}{=} \frac{1}{N_c^2-1} \Big\langle \textsf{Tr}\big\{\textsf{W}(x^-,\vec{x}_\perp) \textsf{W}^{\dagger}(x^-,\vec{y}_\perp)\big\}\Big\rangle_A.
}
\end{fullwidth}
In the {\sffamily MV} model, it may approximately be computed as
\coloredeqnum{glasma49}{
    \textsf{D}_A(x^-,\vec{r}_\perp)\approx\exp{-\frac{r_\perp^2}{4}\textsf{Q}_s^2(x^-,\vec{r}_\perp)},
}
where we introduced the saturation momentum of the gluons
\coloredeqnum{glasma46}{
    \textsf{Q}_s^2(x^-,\vec{r}_\perp)\overset{\Delta}{=}\alpha_sN_c\mu_A(x^-)\ln{\frac{1}{r_\perp^2\Lambda_{\textsf{QCD}}}}.
}

\begin{note}
    We aim to compute the following quantity 
    \begin{align}\label{glasma44}
        \frac{\partial\textsf{D}_A(x^-,\vec{r}_\perp)}{\partial x^-}=\lim_{\epsilon\rightarrow 0}\frac{\textsf{D}_A(x^-+\epsilon,\vec{r}_\perp)-\textsf{D}_A(x^-,\vec{r}_\perp)}{\epsilon}.
    \end{align}
    Let us begin by discretizing the Wilson line along the direction $x^-\overset{\Delta}{=}n\epsilon$, such that the Wilson line at a later $x^-+\epsilon$ is obtained by applying a gauge rotation to the Wilson line at $x^-$ as
    \begin{fullwidth}
    \begin{align*}
        \mathcal{P}\exp{ig\int\limits_{-\infty}^{x^-+\epsilon}dz^-\alpha(z^-,\vec{z}_\perp)}=\mathcal{P}\exp{ig\int\limits_{-\infty}^{x^-}dz^-\alpha(z^-,\vec{z}_\perp)}\exp{ig\int\limits_{x^-}^{x^-+\epsilon}dz^-\alpha(z^-,\vec{z}_\perp)},
    \end{align*}
    \end{fullwidth}
    or equivalently, since $\epsilon$ is an infinitesimal parameter
    \begin{fullwidth}
    \begin{equation}\label{glasma42}
        \begin{aligned}
        \textsf{W}(x^-+\epsilon,\vec{z}_\perp)\approx\textsf{W}(x^-,\vec{z}_\perp)\Big\{\mathds{1}+ig\epsilon\alpha_a(x^-,\vec{z}_\perp)t^a+\frac{1}{2!}(ig)^2\epsilon^2\big[\alpha_a(x^-,\vec{z}_\perp)t^a\big]^2\Big\}.
    \end{aligned}
    \end{equation}
    \end{fullwidth}
    With this discretization, the one and two-point correlators\sidenote{As written in Equations~(\cref{glasma35a}) and~(\cref{glasma35b}).}become\sidenote{For $x^-=n\epsilon$ and $y^-=m\epsilon$, the delta function becomes
    \begin{align*}
        \delta(x^--y^-)=\frac{\delta_{nm}}{\epsilon}.
    \end{align*}
    }
    \vspace*{-0.5\baselineskip}
    \begin{subequations}
    \begin{align}
        &\langle \alpha_a\left(x^-,\vec{x}_\perp\right)\rangle_A=0,\label{glasma43a}\\
        &\langle \alpha_a\left(x^-,\vec{x}_\perp\right)\alpha_b\left(y^-,\vec{y}_\perp\right)\rangle_A=\delta_{ab}\frac{\delta_{nm}}{\epsilon}\lambda_A(x^-)\textsf{L}(\vec{x}_\perp-\vec{y}_\perp).\label{glasma43b}
    \end{align}
    \end{subequations}
    \vspace*{-0.5\baselineskip}
    One may now proceed to evaluating 
    \begin{fullwidth}
    \begin{align*}
        &\textsf{D}_A(x^-+\epsilon,\vec{r}_\perp)\stackrel{(\text{\cref{glasma41}})}{=\joinrel=\joinrel=}\frac{1}{N_c^2-1} \Big\langle \textsf{Tr}\big\{\textsf{W}(x^-+\epsilon,\vec{x}_\perp) \textsf{W}^{\dagger}(x^-+\epsilon,\vec{y}_\perp)\big\}\Big\rangle_A\\
        &\stackrel{(\text{\cref{glasma42}})}{=\joinrel=\joinrel=}\frac{1}{N_c^2-1}\textsf{Tr}\Big\langle\textsf{W}(x^-,\vec{x}_\perp)\Big[\mathds{1}+ig\epsilon\alpha_a(x^-,\vec{x}_\perp)t^a-\frac{g^2}{2}\epsilon^2\alpha_a(x^-,\vec{x}_\perp)t^a\alpha_b(x^-,\vec{x}_\perp)t^b\Big]\times\\
        &\phantom{\stackrel{(\text{\cref{glasma42}})}{=\joinrel=\joinrel=}}\times\Big[\mathds{1}-ig\epsilon\alpha_a(x^-,\vec{y}_\perp)t^a-\frac{g^2}{2}\epsilon^2\alpha_a(x^-,\vec{y}_\perp)t^a\alpha_b(x^-,\vec{y}_\perp)t^b\Big]\textsf{W}^\dagger(x^-,\vec{y}_\perp)\Big\rangle_A\\
        &\stackrel{(\text{\cref{glasma41}})}{=\joinrel=\joinrel=}\textsf{D}_A(x^-,\vec{r}_\perp)\Big\{1+\frac{g^2}{2}\epsilon^2\textsf{Tr}\Big\langle\big[\alpha_a(x^-,\vec{x}_\perp)t^a\big]^2+\big[\alpha_b(x^-,\vec{y}_\perp)t^b\big]^2-2\alpha_a(x^-,\vec{x}_\perp)t^a\alpha_b(x^-,\vec{y}_\perp)t^b\Big\rangle_A\Big\}\\
        &\stackrel{(\text{\cref{glasma43b}})}{=\joinrel=\joinrel=\joinrel=}\textsf{D}_A(x^-,\vec{r}_\perp)\Bigg\{1+\frac{g^2}{2}\epsilon^2\Big[\underbrace{\langle\alpha_a\left(x^-,\vec{x}_\perp\right)\alpha_a\left(x^-,\vec{x}_\perp\right)\rangle_A}_{\lambda_A(x^-)\textsf{L}(\vec{0}_\perp)/\epsilon}(t^a)^2+\underbrace{\langle\alpha_b\left(x^-,\vec{y}_\perp\right)\alpha_b\left(x^-,\vec{y}_\perp\right)\rangle_A}_{\lambda_A(x^-)\textsf{L}(\vec{0}_\perp)/\epsilon}(t^b)^2-\\
        &\phantom{\stackrel{(\text{\cref{glasma43b}})}{=\joinrel=\joinrel=\joinrel=}}-2\underbrace{\langle\alpha_a\left(x^-,\vec{x}_\perp\right)\alpha_b\left(x^-,\vec{y}_\perp\right)\rangle_A}_{\delta_{ab}\lambda_A(x^-)\textsf{L}(\vec{x}_\perp-\vec{y}_\perp)/\epsilon}t^at^b\Big]\Bigg\}=\textsf{D}_A(x^-,\vec{r}_\perp)\Big\{1+\epsilon g^2\underbrace{(t^a)^2}_{N_c}\lambda_A(x^-)\big[\textsf{L}(\vec{0}_\perp)-\textsf{L}(\vec{r}_\perp)\big]\Big\}
    \end{align*}
    \end{fullwidth}
    
    \newpage
    
    This gives\sidenote{By plugging the result obtained above back in Equation~(\cref{glasma44})}a differential equation for the dipole operator
    \begin{align*}
        \frac{\partial\textsf{D}_A(x^-,\vec{r}_\perp)}{\partial x^-}=\textsf{D}_A(x^-,\vec{r}_\perp)\Big\{g^2N_c\lambda_A(x^-)\big[\textsf{L}(\vec{0}_\perp)-\textsf{L}(\vec{r}_\perp)\big]\Big\},
    \end{align*}
    whose solution is given by
    \begin{align*}
        \textsf{D}_A(x^-,\vec{r}_\perp)\stackrel{(\text{\cref{glasma45}})}{=\joinrel=\joinrel=}\mathrm{exp}\Bigg\{-g^2N_c\underbrace{\int\limits_{-\infty}^{x^-}dz^-\lambda_A(z^-)}_{\mu_A(x^-)}\big[\textsf{L}(\vec{0}_\perp)-\textsf{L}(\vec{r}_\perp)\big]\Bigg\}.
    \end{align*}
    Nevertheless, in the small momenta regime, one may compute the term from the exponent as
    \begin{align*}
        \textsf{L}(\vec{0}_\perp)-\textsf{L}(\vec{r}_\perp)&\stackrel{(\text{\cref{glasma37}})}{=\joinrel=\joinrel=}\int \frac{\mathrm{d}^{2} \vec{k}_{\perp}}{(2 \pi)^{2}} \frac{1}{k_{\perp}^{4}}\Big(1-\mathrm{e}^{i \vec{k}_{\perp}\vec{r}_\perp}\Big)\approx\int\frac{\mathrm{d}^2 \vec{k}_\perp}{(2 \pi)^2} \frac{1}{k_\perp^{\cancel{4}2}} \frac{\cancel{k_{\perp}^2} r_\perp^2 \cos ^2 \theta}{2 !}\\
        &=\frac{r_\perp^2}{8\pi^{\cancel{2}1}}\underbrace{\int_0^{\pi}d\theta\cos^2\theta}_{\cancel{\pi}/2}\underbrace{\int\limits_{\Lambda^2}^{1/r_\perp^2}dk_\perp \cancel{k_\perp}\frac{1}{k_\perp^{\cancel{2}1}}}_{\ln{1/(r_\perp^2\Lambda^2)}}=\frac{r_\perp^2}{16\pi}\ln{\frac{1}{r_\perp^2\Lambda^2}}.
    \end{align*}
    This eventually gives\sidenote{After using the definition of the saturation momentum from Equation~(\cref{glasma46}).}
    \begin{align}\label{glasma47}
        \textsf{D}_A(x^-,\vec{r}_\perp)\approx\mathrm{exp}\Bigg\{-\frac{r_\perp^2}{4}\underbrace{\overbrace{\frac{g^2}{4\pi}}^{\alpha_s}N_c\mu_A(x^-)\ln{\frac{1}{r_\perp^2\Lambda^2}}}_{\textsf{Q}_s^2(x^-,\vec{r}_\perp)}\Bigg\}.
    \end{align}
\end{note}

We may finally compute the correlator
\begin{fullwidth}
\begin{align}\label{glasma50}
    \langle F_{a}^{i+}(\vec{k}) F_{a}^{i+}(-\vec{k})\rangle_A=\frac{2S_\perp C_F}{\alpha_s\pi}\int\mathrm{d}^2\vec{r}_\perp\frac{\mathrm{e}^{i\vec{k}_\perp\vec{r}_\perp}}{r_\perp^2}\Big(1-\exp{-\frac{r_\perp^2}{4}\textsf{Q}_s^2(r_\perp)}\Big).
\end{align}
\end{fullwidth}

\begin{note}
    Let us successively collect all of the obtained results and write
    
    \newpage
    
    \begin{fullwidth}
    \begin{align*}
        &\langle F_{a}^{i+}(\vec{k}) F_{a}^{i+}(-\vec{k})\rangle_A\stackrel{(\text{\cref{glasma32}})}{=\joinrel=\joinrel=}\int \mathrm{d}^{3} x \int \mathrm{d}^{3} y \text{  }\mathrm{e}^{i(\vec{x}-\vec{y}) \cdot \vec{k}} \langle F^{i+}_a(\vec{x})F^{i+}_a(\vec{y})\rangle_A\\
        &\underset{(\text{\cref{glasma41}})}{\stackrel{(\text{\cref{glasma48}})}{=\joinrel=\joinrel=}}\int dx^-\int \mathrm{d}^{2} \vec{x}_\perp \int \mathrm{d}^{2} \vec{y}_\perp \text{  }\mathrm{e}^{i(\vec{x}_\perp-\vec{y}_\perp) \cdot \vec{k}_\perp}\frac{\partial\mu_A(x^-)}{\partial x^-}\frac{1}{4\pi}\ln{\frac{1}{r_\perp^2\Lambda_{\textsf{QCD}}^2}}(N_c^2-1)\textsf{D}_A(x^-,\vec{r}_\perp)\\
        &\underset{(\text{\cref{glasma46}})}{\stackrel{(\text{\cref{glasma49}})}{=\joinrel=\joinrel=}}\frac{N_c^2-1}{4\pi}\underbrace{\int \mathrm{d}^{2} \vec{x}_\perp}_{S_\perp}\int \mathrm{d}^{2} \vec{r}_\perp \text{  }\mathrm{e}^{i\vec{r}_\perp \cdot \vec{k}_\perp}\int dx^-\frac{\partial\mu_A(x^-)}{\partial x^-}\ln{\frac{1}{r_\perp^2\Lambda_{\textsf{QCD}}^2}}\exp{-\frac{r_\perp^2}{4}\alpha_sN_c\mu_A(x^-)\ln{\frac{1}{r_\perp^2\Lambda_{\textsf{QCD}}}}}
    \end{align*}
    \end{fullwidth}
    By performing the change of variables
    \begin{align*}
        \widetilde{\mu}_A(x^-)\overset{\Delta}{=}1-\exp{-\frac{r_\perp^2}{4}\alpha_sN_c\mu_A(x^-)\ln{\frac{1}{r_\perp^2\Lambda_{\textsf{QCD}}}}},
    \end{align*}
    we immediately obtain
    \begin{fullwidth}
    \begin{align*}
        \frac{\partial\widetilde{\mu}_A(x^-)}{\partial x^-}=-\frac{r_\perp^2}{4}\alpha_sN_c\frac{\partial\mu_A(x^-)}{\partial x^-}\ln{\frac{1}{r_\perp^2\Lambda_{\textsf{QCD}}^2}}\exp{-\frac{r_\perp^2}{4}\alpha_sN_c\mu_A(x^-)\ln{\frac{1}{r_\perp^2\Lambda_{\textsf{QCD}}}}}
    \end{align*}
    \end{fullwidth}
    which after simple manipulations leads to
    \begin{fullwidth}
    \begin{align*}
        &\langle F_{a}^{i+}(\vec{k}) F_{a}^{i+}(-\vec{k})\rangle_A=\frac{N_c^2-1}{\cancel{4}\pi}S_\perp\int \mathrm{d}^{2} \vec{r}_\perp \text{  }\mathrm{e}^{i\vec{r}_\perp \cdot \vec{k}_\perp}\frac{\cancel{4}}{r_\perp^2\alpha_sN_c}\underbrace{\int dx^-\frac{\partial\widetilde{\mu}_A(x^-)}{\partial x^-}}_{\widetilde{\mu}_A(x^-)}\\
        &=\frac{2S_\perp}{\alpha_s\pi}\underbrace{\frac{N_c^2-1}{2N_c}}_{C_F}\int\mathrm{d}^2\vec{r}_\perp\frac{\mathrm{e}^{i\vec{k}_\perp\vec{r}_\perp}}{r_\perp^2}\Big(1-\exp{-\frac{r_\perp^2}{4}\alpha_sN_c\mu_A(x^-)\ln{\frac{1}{r_\perp^2\Lambda_{\textsf{QCD}}}}}\Big)
    \end{align*}
    \end{fullwidth}
\end{note}

This enables us to evaluate the gluon occupation number, as obtained from the {\sffamily MV} model\sidenote{With $r_\perp<1/\Lambda_{\textsf{QCD}}$, such that the quantity $\ln{1/r_\perp^2\Lambda_{\textsf{QCD}}^2}$ from the exponent doesn't change the sign.}
\coloredeqnum{glasma51}{
    f(k_\perp)\underset{(\text{\cref{glasma50}})}{\stackrel{(\text{\cref{glasma29}})}{=\joinrel=\joinrel=\joinrel=}}\int\mathrm{d}^2\vec{r}_\perp\mathrm{e}^{i\vec{k}_\perp\vec{r}_\perp}\frac{1-\exp{-\frac{r_\perp^2}{4}\textsf{Q}_s^2(r_\perp)}}{\pi\alpha_SN_cr_\perp^2}.
}
Two regimes distinguish themselves, namely one in which the exponent is higher than unity, where saturation occurs, and one where the exponent is sub-unitary, known as the dilute regime. It will turn out useful to define 
\begin{align*}
    \textsf{Q}_A^2\overset{\Delta}{=}\alpha_sN_c\mu_A\stackrel{(\text{\cref{glasma7}})}{=\joinrel=\joinrel=}\alpha_sN_c\frac{A}{\cancel{2\pi} R_A^2}\underbrace{\cancelto{2}{4\pi}\alpha_s}_{g^2}=\frac{2\alpha_s^2N_cA}{R_A^2}\sim A^{1/3},
\end{align*}
such that the saturation momentum\sidenote{Defined in Equation~(\cref{glasma46}).}may be rewritten as
\begin{align*}
    \textsf{Q}_s^2(A,r_\perp)=\textsf{Q}_A^2\ln{\frac{1}{r_\perp^2\Lambda_{\textsf{QCD}}}}.
\end{align*}
The {\sffamily\color{ming} saturation momentum} is the scale which separates the two regimes and it is customary defined as the value at which, for 
\begin{align*}
    \textsf{Q}_s(A)\overset{\Delta}{=}\textsf{Q}_s\Big(A,r_\perp\overset{\Delta}{=}\frac{2}{\textsf{Q}_s(A)}\Big),
\end{align*}
the exponent reaches unity, that is
\begin{align*}
    \left.-\frac{r_\perp^2}{4}\textsf{Q}_A^2\ln{\frac{1}{r_\perp^2\Lambda_{\textsf{QCD}}}}\right|_{r_\perp=2/\textsf{Q}_s(A)}\approx 1.
\end{align*}
This yields the saturation scale
\coloredeq{
    \textsf{Q}_s^2(A)\approx \textsf{Q}_A^2\ln{\frac{\textsf{Q}_A^2}{\Lambda_{\textsf{QCD}}^2}}\sim A^{1/3}\ln{A^{1/3}}.
}
With respect to this saturation scale, the gluon occupation factor\sidenote{From Equation~(\cref{glasma51})}behaves as
\begin{fullwidth}
\begin{align*}
    f(k_\perp)\approx\begin{cases}
    \dfrac{1}{\alpha N_{c}} \dfrac{\textsf{Q}_{A}^{2}}{k_{\perp}^{2}}\left\{1+\dfrac{\textsf{Q}_{A}^{2}}{k_{\perp}^{2}}\left[\ln \dfrac{k_{\perp}^{2}}{\Lambda_{\textsf{OCD}}^{2}}+2 \gamma_{\textsf{E}}-2\right]\right\}\sim \dfrac{A^{1/3}}{k_\perp^2}, &\text { for }k_{\perp} \gg \textsf{Q}_{s}(A)\\
    \vspace*{-5pt} \\
    \dfrac{1}{\alpha N_{c}} \ln \dfrac{\textsf{Q}_{s}^{2}(A)}{k_{\perp}^{2}}\sim\ln{\dfrac{A^{1/3}}{k_\perp^2}}, &\text { for }k_{\perp} \ll \textsf{Q}_{s}(A)
    \end{cases}
\end{align*}
\end{fullwidth}
In the dense limit, this factor grows only logarithmically and is thus said to reach a saturation regime.

% \begin{note}
%     In the dilute regime, where the transverse momenta are much larger that the saturation scale\sidenote{Or equivalently, the dipole has a transverse size $r_\perp\ll 1/\textsf{Q}_s(A)$.}$k_\perp\gg\textsf{Q}_s(A)$, one may perform the series expansion of the dipole operator\sidenote{While neglecting the logarithmic dependence of $\textsf{Q}_s^2(r_\perp)$ on $r_\perp$, see Equation~(\cref{glasma46}).}
%     \begin{align*}
%         f(k_\perp)\approx\frac{1}{\pi\alpha_sN_c}\int\mathrm{d}^2\vec{r}_\perp\frac{\mathrm{e}^{i\vec{k}_\perp\vec{r}_\perp}}{r_\perp^2}\left(\frac{r_\perp^2 \textsf{Q}_s^2}{4}-\frac{1}{2} \frac{r_\perp^4 \textsf{Q}_s^2}{16}\right)
%     \end{align*}
% \end{note}

